\Block{Polyphase FIR Filter (frontend for a full PFB)}
{pfb\_fir}
{pfbfir}
{Aaron Parsons}
{Aaron Parsons}
{This block, combined with an FFT, implements a Polyphase Filter Bank which uses longer windows of data to improve the shape of channels within a spectrum.}

\begin{ParameterTable}
\Parameter{Size of PFB: ($2^?$)}{PFBSize}{The number of channels in the PFB (this should also be the size of the FFT which follows).}
\Parameter{Total Number of Taps:}{TotalTaps}{The number of taps in the PFB FIR filter.  Each tap uses 2 real multiplier cores and requires buffering the real and imaginary streams for $2^{PFBSize}$ samples.}
\Parameter{Windowing Function}{WindowType}{Which windowing function to use (this allows trading passband ripple for steepness of rolloff, etc).}
\Parameter{Number of Simultaneous Inputs: ($2^?$)}{n\_inputs}{The number of parallel time samples which are presented to the FFT core each clock.  The number of output ports are set to this same value.}
\Parameter{Make Biplex}{MakeBiplex}{Double up the inputs to match with a biplex FFT.}
\Parameter{Input Bit Width}{BitWidthIn}{The number of bits in each real and imaginary sample input to the PFB.}
\Parameter{Output Bit Width}{BitWidthOut}{The number of bits in each real and imaginary sample output from the PFB.  This should match the bit width in the FFT that follows.}
\Parameter{Coefficient Bit Width}{CoeffBitWidth}{The number of bits in each coefficient.  This is usually chosen to match the input bit width.}
\Parameter{Use Distributed Memory for Coefficients}{CoeffDistMem}{Store the FIR coefficients in distributed memory (if = 1).  Otherwise, BRAMs are used to hold the coefficients.}
\Parameter{Add Latency}{add\_latency}{Latency through adders in the FFT.}
\Parameter{Mult Latency}{mult\_latency}{Latency through multipliers in the FFT.}
\Parameter{BRAM Latency}{bram\_latency}{Latency through BRAM in the FFT.}
\Parameter{Quantization Behavior}{quantization}{Specifies the rounding behavior used at the end of each butterfly computation to return to the number of bits specified above.}
\Parameter{Bin Width Scaling (normal = 1)}{fwidth}{PFBs give enhanced control over the width of frequency channels.  By adjusting this parameter, you can scale bins to be wider (for values > 1) or narrower (for values < 1).}
\end{ParameterTable}

\begin{PortTable}
\Port{sync}{in}{Boolean}{Indicates the next clock cycle contains valid data}
\Port{pol\_in}{in}{Inherited}{The (complex) time-domain stream(s).}
\Port{sync\_out}{out}{Boolean}{Indicates that data out will be valid next clock cycle.}
\Port{pol\_out}{out}{Inherited}{The (complex) PFB FIR output, which is still a time-domain signal.}
\end{PortTable}

\BlockDesc{This block, combined with an FFT, implements a Polyphase Filter Bank which uses longer windows of data to improve the shape of channels within a spectrum.}
