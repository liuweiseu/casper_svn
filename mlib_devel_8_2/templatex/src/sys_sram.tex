%%%%%%%%%%%%%%%%%%%%%%%%%%%%%%%%%
%%%SRAM Block
%%%%%%%%%%%%%%%%%%%%%%%%%%%%%%%%%



\Block{SRAM}{sram}{sram}{Pierre Yves Droz, Henry Chen}{Ben Blackman}{The sram block represents a 36x512k SRAM chip on the IBOB. It stores 36-bit words and requires 19 bits to access its address space.}


\begin{ParameterTable}
\Parameter{SRAM}{sram}{Selects which SRAM chip this block represents.}
\Parameter{Data Type}{arith\_type}{Type to which the data is cast on both the input and output.}
\Parameter{Data binary point (bitwidth is 36)}{bin\_pt}{Position of the binary point of the data.}
\Parameter{Sample period}{sample\_period}{Sets the period with reference to the clock frequency.}
\Parameter{Simulate SRAM using ModelSim}{use\_sim}{Turns ModelSim simulation on or off.}
\end{ParameterTable}

\begin{PortTable}
\Port{we}{IN}{boolean}{A signal that when high, causes the data on data\_in to be written to address.}
\Port{be}{IN}{unsigned\_4\_0}{A signal that enables different 9-bit bytes of data\_in to be written.}
\Port{address}{IN}{unsigned\_19\_0}{A signal that specifies the address where either data\_in is to be stored or from where data\_out is to be read.}
\Port{data\_in}{IN}{arith\_type\_36}{A signal that contains the data to be stored.}
\Port{data\_out}{OUT}{arith\_type\_36}{A signal that contains the data coming out of address.}
\Port{data\_valid}{OUT}{boolean}{A signal that is high when data\_out is valid.}
\end{PortTable}

\BlockDesc{
\paragraph{Usage}
The SRAM block is 36x512k, signifying that its input and output are 36-bit words and it can store 512k words. Each clock cyle, if \textit{we} is high, then each bit of be determines whether each 9-bit chunk will be written to address. \textit{be} is 4 bits with the highest bit corresponding to the most significant chunk (so if \textit{be} is 1100, only the top 18 bits will be written). If \textit{we} is low, then the SRAM block ignores \textit{data\_in} and be and reads the word stored at \textit{address}.
}
