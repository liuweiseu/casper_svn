\documentclass{article}
\oddsidemargin  0.0in
\evensidemargin 0.0in
\textwidth      6.5in
\usepackage{tabularx}
\usepackage{html}
\title{\textbf{CASPER Library} \\Reference Manual}
\newcommand{\Block}[6]{\section {#1 \emph{(#2)}} \label{#3} \textbf{Block Author}: #4 \\ \textbf{Document Author}: #5 \subsection*{Summary}#6}

\newenvironment{PortTable}{\subsection*{Ports}
\tabularx{6.5in}{|l|l|l|X|} \hline  \textbf{Port} & \textbf{Dir.} & \textbf{Data Type} & \textbf{Description} \\ \hline}{\endtabularx}

\newcommand{\Port}[4]{\emph{#1} & \lowercase{#2} & #3 & #4\\  \hline}

\newcommand{\BlockDesc}[1]{\subsection*{Description}#1}

\newenvironment{ParameterTable}{\subsection*{Mask Parameters}
\tabularx{6.5in}{|l|l|X|} \hline  \textbf{Parameter} & \textbf{Variable} & \textbf{Description} \\ \hline}{\endtabularx}

\newcommand{\Parameter}[3]{#1 & \emph{#2} & #3 \\ \hline}

\begin{htmlonly}
\newcommand{\tabularx}[3]{\begin{tabularx}{#1}{#2}{#3}}
\newcommand{\endtabularx}{\end{tabularx}}
\end{htmlonly}

\date{Last Updated \today}
\begin{document}
\maketitle

%\chapter{System Blocks}
%%%%Change Chapter%%%%%%%%
%\chapter{Signal Processing Blocks}

%\input{test.tex}
%\chapter{Communication Blocks}
%\end{document} 
\Block{Real-sampled Biplex FFT (with output demuxed by 2)}
{fft\_biplex\_real\_2x}
{fftbiplexreal2x}
{Aaron Parsons}
{Aaron Parsons}
{Computes the real-sampled Fast Fourier Transform using the standard Hermitian conjugation trick to use a complex core to transform a two real streams.  Thus, a biplex core (which can do 2 complex FFTs) can transform 4 real streams.  Only positive frequencies are output (negative frequencies are the mirror images of their positive counterparts).  Data is output in normal frequency order, meaning that channel 0 (corresponding to DC) is output first, followed by channel 1, on up to channel $2^{N-1}-1$.  Real inputs 1 and 2 share one output port (with the data for 1 coming first, then the data for 2), and likewise for inputs 3 and 4.}

\begin{ParameterTable}
\Parameter{Size of FFT: ($2^?$)}{FFTSize}{The number of channels computed in the complex FFT core.  The number of channels output for each real stream is half of this.}
\Parameter{Bit Width}{BitWidth}{The number of bits in each real and imaginary sample as they are carried through the FFT.  Each FFT stage will round numbers back down to this number of bits after performing a butterfly computation.}
\Parameter{Quantization Behavior}{quantization}{Specifies the rounding behavior used at the end of each butterfly computation to return to the number of bits specified above.}
\Parameter{Overflow Behavior}{overflow}{Indicates the behavior of the FFT core when the value of a sample exceeds what can be expressed in the specified bit width.}
\Parameter{Add Latency}{add\_latency}{Latency through adders in the FFT.}
\Parameter{Mult Latency}{mult\_latency}{Latency through multipliers in the FFT.}
\Parameter{BRAM Latency}{bram\_latency}{Latency through BRAM in the FFT.}

\end{ParameterTable}

\begin{PortTable}
\Port{sync}{in}{Boolean}{Indicates the next clock cycle contains valid data}
\Port{shift}{in}{Unsigned}{Sets the shifting schedule through the FFT.  Bit 0 specifies the behavior of stage 0, bit 1 of stage 1, and so on.  If a stage is set to shift (with bit = 1), that every sample is divided by 2 at the output of that stage.}
\Port{pol}{in}{Inherited}{The time-domain stream(s) to channelized.}
\Port{sync\_out}{out}{Boolean}{Indicates that data out will be valid next clock cycle.}
\Port{of}{out}{Boolean}{Indicates an overflow occurred at some stage in the FFT.}
\Port{pol\_out}{out}{Inherited}{The frequency channels.}
\end{PortTable}

\BlockDesc{Computes the real-sampled Fast Fourier Transform using the standard Hermitian conjugation trick to use a complex core to transform a two real streams.  Thus, a biplex core (which can do 2 complex FFTs) can transform 4 real streams.  Only positive frequencies are output (negative frequencies are the mirror images of their positive counterparts).  Data is output in normal frequency order, meaning that channel 0 (corresponding to DC) is output first, followed by channel 1, on up to channel $2^{N-1}-1$.  Real inputs 1 and 2 share one output port (with the data for 1 coming first, then the data for 2), and likewise for inputs 3 and 4.}
 
\end{document}
