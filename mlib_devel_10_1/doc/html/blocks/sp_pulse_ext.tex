\documentclass{article}
\oddsidemargin  0.0in
\evensidemargin 0.0in
\textwidth      6.5in
\usepackage{tabularx}
\usepackage{html}
\title{\textbf{CASPER Library} \\Reference Manual}
\newcommand{\Block}[6]{\section {#1 \emph{(#2)}} \label{#3} \textbf{Block Author}: #4 \\ \textbf{Document Author}: #5 \subsection*{Summary}#6}

\newenvironment{PortTable}{\subsection*{Ports}
\tabularx{6.5in}{|l|l|l|X|} \hline  \textbf{Port} & \textbf{Dir.} & \textbf{Data Type} & \textbf{Description} \\ \hline}{\endtabularx}

\newcommand{\Port}[4]{\emph{#1} & \lowercase{#2} & #3 & #4\\  \hline}

\newcommand{\BlockDesc}[1]{\subsection*{Description}#1}

\newenvironment{ParameterTable}{\subsection*{Mask Parameters}
\tabularx{6.5in}{|l|l|X|} \hline  \textbf{Parameter} & \textbf{Variable} & \textbf{Description} \\ \hline}{\endtabularx}

\newcommand{\Parameter}[3]{#1 & \emph{#2} & #3 \\ \hline}

\begin{htmlonly}
\newcommand{\tabularx}[3]{\begin{tabularx}{#1}{#2}{#3}}
\newcommand{\endtabularx}{\end{tabularx}}
\end{htmlonly}

\date{Last Updated \today}
\begin{document}
\maketitle

%\chapter{System Blocks}
%%%%Change Chapter%%%%%%%%
%\chapter{Signal Processing Blocks}

%\input{test.tex}
%\chapter{Communication Blocks}
%\end{document} 
\Block{The Pulse Extender Block}{pulse\_ext}{pulseext}{Aaron Parsons}{Aaron Parsons}{Extends a boolean signal to be high for the specified number of clocks after the last high input.}



\begin{ParameterTable}

\Parameter{Length of Pulse}{pulse\_len}{Specifies number of clocks after the last high input for which the output is held high.}

\end{ParameterTable}



\begin{PortTable}

\Port{in}{in}{Boolean}{Input boolean signal.}

\Port{out}{out}{Boolean}{Pulse-extended boolean signal.}

\end{PortTable}



\BlockDesc{Extends a boolean signal to be high for the specified number of clocks after the last high input. If a new in pulse (input high) occurs, the counter determining the output pulse length is reset.} 
\end{document}
