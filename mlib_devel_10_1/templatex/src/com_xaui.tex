
\Block{XAUI Transceiver}{XAUI}{XAUI}
{Pierre Yves Droz, Henry Chen}
{Jason Manley}
{XAUI block for sending and receiving point-to-point, streaming data over the BEE2 and iBOB's CX4 connectors. NOTE: A new version of this block is in development.}


\begin{ParameterTable}
\Parameter{Demux}{demux}{Selects the width of the data bus. 1 for 64 bits, 2 for 32 bits.}
\Parameter{Port}{port}{Selects the physical CX4 port on the iBOB or BEE2. The iBOB has two ports; the BEE2 has two for the control FPGA and four for each of the user FPGAs. CORR is not used by CASPER.}
\Parameter{Pre-emphasis}{pre\_emph}{Selects the pre-emaphasis to use over the physical link. Default: 3 (see Xilinx documentation)}
\Parameter{Differential Swing}{swing}{Selects the size of the differential swing to use in mV. Default: 800 (see Xilinx documentation)}
\end{ParameterTable}

\begin{PortTable}
\Port{rx\_get}{in}{boolean}{Used to request the next data word from the RX buffer.}
\Port{rx\_reset}{in}{boolean}{Resets the receive subsystem.}
\Port{tx\_data}{in}{ufix\_64\_0 or ufix\_32\_0}{Accepts the next data word (64 or 32 bits) to be transmitted.}
\Port{tx\_outofband}{in}{ufix\_8\_0 or ufix\_4\_0}{Accepts the next data word (8 bits if demux=1, 4 bits if demux=2) to be transmitted through the out-of-band channel.}
\Port{tx\_valid}{out}{boolean}{Clocks the transmit data into the transceiver. Data is clocked into the buffer while this line is high.}
\Port{rx\_data}{out}{ufix\_64\_0}{Outputs the received data stream.}
\Port{rx\_outofband}{out}{ufix\_8\_0 or ufix\_4\_0}{Outputs the out-of-band received data stream.}
\Port{rx\_empty}{out}{boolean}{Indicates that the receive buffer is empty.}
\Port{rx\_valid}{out}{boolean}{Indicates that data has been received.}
\Port{rx\_linkdown}{out}{boolean}{Indicates that the link is down (eg. faulty cable).}
\Port{tx\_full}{out}{boolean}{Indicates the transmit buffer is full.}
\Port{rx\_almost\_full}{boolean}{inherited}{Indicates the receive buffer is full.}
\end{PortTable}

\BlockDesc{This block is due to be deprecated soon. It will be replaced by a new XAUI block in the CASPER library.\\
\paragraph{Demux}
Perhaps a misnomer, this parameter describes the width of the data bus rather than a selection of two muxed streams on one channel. At 156MHz XAUI clock, the maximum transmission speed is 64bits * 156.25 MHz = 10Gbit/s. For BEE or iBOB designs clocked at rates above 156MHz, clocking-in 64 bit data on every clock cycle would cause the XAUI block's FIFO buffers to overflow. The \textit{demux} option is provided which halves the input data bus width to 32 bits and enables data to be clocked-in on every FPGA clock cycle. Along with the data bus width, the \textit{out of band} bus width is also halved to 4 bits.

\paragraph{Out of band signals}
Out of band signals are guaranteed to arrive at the same time as the data word with which they were sent. Out-of-band data is only transmitted across the physical link if the input to \textit{tx\_outofband} changes and is clocked in as valid (\textit{tx\_valid}). In other words, if you keep \textit{tx\_outofband} constant, no additional bandwidth is consumed (the in-band signals are transmitted as normal). When data is clocked into the transmitter, it will appear out the receiver as if the \textit{tx\_outofband} and \textit{tx\_data} arrived simultaneously. Care should be taken to ensure that the data clocked into \textit{tx\_outofband} and \textit{tx\_data} does not exceed the XAUI's maximum transmission rate (approximately 10Gbps for 156.25MHz clock). Each change of \textit{tx\_outofband} (be it one bit or eight bits) requires 64 bits (a full word) to transmit. This bus width is 8 bits if \textit{demux} is not selected (set to 1), and 4 bits if it is set to 2.
}