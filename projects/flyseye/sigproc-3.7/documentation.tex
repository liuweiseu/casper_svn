\documentclass[11pt]{article}
\usepackage{makeidx}
\makeindex
\def\lapp{\ifmmode\stackrel{<}{_{\sim}}\else$\stackrel{<}{_{\sim}}$\fi}
\def\gapp{\ifmmode\stackrel{>}{_{\sim}}\else$\stackrel{>}{_{\sim}}$\fi}
\textheight 670pt 
\textwidth 500pt  
\evensidemargin 5mm            
\oddsidemargin -5mm  
\topmargin -15mm     
\headheight 12pt 
\headsep 15pt
\parindent 0pt
\parskip 8pt plus 2pt minus 1pt

\begin{document}
\begin{center}
{\LARGE {\sc SIGPROC--vX.X} : {\bf (Pulsar) Signal Processing Programs}}\\
\bigskip
\bigskip
{\large Dunc Lorimer --- Jodrell Bank Observatory --- {\tt drl@jb.man.ac.uk} --- RELEASE}
\end{center}
\noindent {\bf Summary:} The SIGPROC package is a collection of
programs written to convert and process fast-sampled pulsar data into
a compact and easy-to-use format suitable for off-line analyses for
searching, timing and polarimetry applications. This document
describes how to install and run the various programs. Several example
applications are presented using real and simulated data sets.
\tableofcontents

\clearpage
\section{About SIGPROC}

\index{Backends!AOFTM}
\index{Backends!WAPP}
\index{Backends!PSPM}
\index{Backends!BPP}
\index{Backends!Parkes/Jodrell filterbanks}
\index{Backends!OOTY}
SIGPROC is a package designed to standardize the initial analysis of
the many types of fast-sampled pulsar data. Currently recognized
machines are the Wide Band Arecibo Pulsar Processor (WAPP), the Penn
State Pulsar Machine (PSPM), the Arecibo Observatory Fourier Transform
Machine (AOFTM), the Berkeley Pulsar Processors (BPP), the Parkes/Jodrell
1-bit filterbanks (SCAMP) and the
filterbank at the Ooty radio telescope (OOTY). The package
should help users look at their data quickly, without the need to
write (yet) another routine to read data or worry about big/little
\index{byte swapping} \index{big endian} \index{little endian}
endian compatibility (byte swapping is handled automatically).  The
current suite of programs is:

\index{Programs!{\tt filterbank}}
\bigskip
\noindent {\tt filterbank} - convert raw pulsar-machine data to 
filterbank format: a stream of n-bit numbers corresponding to multiple
polarization and/or frequency channels. 

\index{Programs!{\tt splice}}
\smallskip
\noindent {\tt splice} - join together multiple filterbank
files which have the same time stamp.

\index{Programs!{\tt fake}}
\smallskip
\noindent {\tt fake} - produce fake filterbank format data
containing periodic signals immersed in Gaussian noise for
testing and calibration of downstream programs.

\index{Programs!{\tt decimate}}
\smallskip
\noindent {\tt decimate} - add together frequency channels
and/or time samples of incoming filterbank data to reduce
the time and/or frequency resolution (useful for quick-look
purposes).

\index{Programs!{\tt dedisperse}}
\smallskip
\noindent {\tt dedisperse} - correct incoming filterbank data
for interstellar dispersion, writing the output time series as
one or more dedispersed sub-bands.

\index{Programs!{\tt fold}}
\smallskip
\noindent {\tt fold} - fold incoming filterbank or time series
data modulo a pulse period. Pulses are output in ASCII
EPN (\S \ref{epn}) or PSRFITS format. An Expect script to 
generate polynomial coefficients is also available.

\index{Programs!{\tt profile}}
\smallskip
\noindent {\tt profile} - displays profiles from {\tt fold} in
ASCII or pseudo grey-scale plots to the standard output.

\index{Programs!{\tt pgplotter}}
\smallskip
\noindent {\tt pgplotter} - displays profiles from {\tt fold} 
and other SIGPROC output to a PGPLOT window.

\index{Programs!{\tt bandpass}}
\smallskip
\noindent {\tt bandpass} - write out the mean bandpass to an ASCII file.

\index{Programs!{\tt header}}
\smallskip
\noindent {\tt header} - read raw data files,
filterbank or time series data and display header info as plain ASCII.

\index{Programs!{\tt header}}
\smallskip
\noindent {\tt reader} - read the filterbank or time series
data and display in human-readable form.

\index{Programs!{\tt quicklook}}
\smallskip
\noindent {\tt quicklook} - csh script to perform a quick
analysis of total-power filterbank data on a known pulsar.

\index{Programs!{\tt monitor}}
\smallskip
\noindent {\tt monitor} - wish script to
monitor programs running in a given
directory using a Tk pop-up widget.

\bigskip
\noindent
All of the programs within SIGPROC are written in C and can be run
from the UNIX command-line. Use is made of standard input and output
streams so that piping between programs is possible to ``glue''
together various tasks. As an example, the following pipeline:
\begin{verbatim}
% filterbank B0823+26.pspm | dedisperse -d 19 -s 4 | fold -p polyco.dat > B0823+26.prf
\end{verbatim}
will read in and dedisperse raw PSPM data into four subbands which are
then folded modulo the pulse period based on a set of polynomial
coefficients generated by TEMPO stored in the file {\tt polyco.dat}.
The folded profiles for each band are written in ASCII format
to the file {\tt B0823+26.prf}.  
\index{polyco.dat}

A detailed description of these programs and scripts is given in the
remainder of this document which is structured as follows: in \S
\ref{install} we describe how to install SIGPROC; \S \ref{dataformat}
describes the filterbank data and header format used by all the
programs; producing real and fake filterbank data is described in \S
\ref{filterbank} and \S \ref{fake} respectively; programs to look at
the headers and raw data are discussed in \S \ref{headers} and \S
\ref{looking}; data reduction tasks (decimation and dedispersion) are
described in \S \ref{reduction}; folding filterbank data to produce
pulse profiles is described in \ref{folding}; a script do quick data
analyses is presented in \S \ref{quicklook} respectively; version
history and plans for future work (\S \ref{past/future}). 
Supplementary appendices deal with
monitoring the programs (\S \ref{monitoring}), 
generating {\tt polyco.dat} files using {\sc TEMPO}
(\S \ref{polyco}) and the EPN data format (\S \ref{epn}).

\section{Installation procedure}
\label{install}
\index{installation}
SIGPROC has so far been successfully installed for use on Solaris,
Linux, HP-UX and Macs.
ANSII C was (hopefully!)  adhered to fairly closely
during writing of the programs so that installation on other operating
systems should also be possible.  Installation proceeds as follows:

\noindent {\bf 0:}
Download the package from 
\verb+http://www.jb.man.ac.uk/~drl/sigproc/sigproc.tar.gz+

\noindent {\bf 1:}
Unpack the gzip-compressed tar file and extract its contents:

\noindent 
{\tt gunzip -c sigproc-X.X.tar.gz | tar xvf -}

\noindent {\bf 2:}
The contents of the tar file will be distributed in the directory
{\tt sigproc-X.X/}. Go into this directory and run the configuration
script by typing: 

\noindent
{\tt cd sigproc-X.X}\\
{\tt ./configure}

\noindent
When prompted, supply the name
of a directory in which you would like the SIGPROC executables to
be placed.
If compiling on more than one system, log into the the other system
and run the same script on this computer.
Note that only one copy of the source code is required
if you are compiling under multiple platforms.

\noindent {\bf 3:}
For each operating system you are using, type:

\noindent
{\tt make}

\noindent in the
{\tt sigproc-X.X} directory
and let the compiler go to work. 

Four other software packages are desirable, but not absolutely
necessary.  To output profiles in PSRFITS format, you will need
CFITSIO ({\tt 
heasarc.gsfc.nasa.gov/docs/software/fitsio/fitsio.html})
and uncomment and edit the appropriate path to the LFITS variable in your
{\tt makfile.osname} file. To take advantage of the FFTW subroutines,
you will need to install version 3 of this package (available from
{\tt fftw.org}) and then uncomment and edit the LFFTW variable in
{\tt makfile.osname}.
To create files containing polynomial coefficients for
high-precision folding, install the {\sc TEMPO} software package which
\index{Software Packages!{\sc TEMPO}}
is freely available from the Princeton pulsar website ({\tt
pulsar.princeton.edu}).  To monitor the programs using a Tk
pop-up widget make sure that the {\tt wish} shell is in your path (we
recommend use of Tcl/Tk version 8.0 or higher). This is freely
available from {\tt scriptics.com}. For making diagnostic
\index{Software Packages!Tcl/Tk}
plots you will need to compile the {\tt quickplot} Fortran program which
requires the {\sc PGPLOT} graphics package available from 
\verb+astro.caltech.edu/~tjp/pgplot+. Edit the {\tt
makefile} to give the appropriate path to {\sc PGPLOT} on your system
\index{Software Packages!{\sc PGPLOT}}
before typing {\tt make quickplot}.

\clearpage
\section{Header information and data format}
\label{dataformat}
\index{Data formats!SIGPROC}
Before describing the programs in detail, some description of the
header and data formats used within SIGPROC is appropriate for those
wishing to read the data into other programs.  The {\tt filterbank}
program (see \S \ref{filterbank}) reads in the raw data files produced
by the machine, dealing with the header information contained in the
files and the (usually non-trivial) channel ordering of the
samples. {\tt filterbank} outputs the data in the following way:
\begin{verbatim}
HEADER_START stream_of_header_parameters HEADER_END stream_of_data_values
\end{verbatim}
The \verb+HEADER_START+ and \verb+HEADER_END+ character strings
signal the start and
finish of a stream of header parameters that describe the data.  The
default is to include these at the beginning of the data file.  We
recognize that some users will prefer not to have to deal with the
header in this way. For these users, {\tt filterbank} has a {\tt
-headerfile} command-line option to pipe the header into a seperate
ASCII file (this is described along with the other command-line
options later on).

The header variables have been restricted to key parameters for ease of use.
Currently these are:
\begin{itemize}
\item {\bf telescope\_id} (\verb+int+): 
\index{Header parameters!{\bf telescope\_id}}
0=fake data; 1=Arecibo; 2=Ooty... others to be added
\item {\bf machine\_id} (\verb+int+): 
\index{Header parameters!{\bf machine\_id}}
0=FAKE; 1=PSPM; 2=WAPP; 3=OOTY... others to be added
\item {\bf data\_type} (\verb+int+): 
\index{Header parameters!{\bf data\_type}}
1=filterbank; 2=time series... others to be added
\item {\bf rawdatafile} (\verb+char []+): 
\index{Header parameters!{\bf rawdatafile}}
the name of the original data file
\item {\bf source\_name} (\verb+char []+): 
\index{Header parameters!{\bf source\_name}}
the name of the source being observed by the telescope
\item {\bf barycentric} (\verb+int+):
equals 1 if data are barycentric or 0 otherwise
\item {\bf pulsarcentric} (\verb+int+):
equals 1 if data are pulsarcentric or 0 otherwise
\item {\bf az\_start} (\verb+double+): 
\index{Header parameters!{\bf az\_start}}
telescope azimuth at start of scan (degrees)
\item {\bf za\_start} (\verb+double+): 
\index{Header parameters!{\bf za\_start}}
telescope zenith angle at start of scan (degrees)
\item {\bf src\_raj} (\verb+double+): 
\index{Header parameters!{\bf src\_raj}}
right ascension (J2000) of source (hhmmss.s)
\item {\bf src\_dej} (\verb+double+): 
\index{Header parameters!{\bf src\_dej}}
declination (J2000) of source (ddmmss.s)
\item {\bf tstart} (\verb+double+): 
\index{Header parameters!{\bf tstart}}
time stamp (MJD) of first sample
\item {\bf tsamp}  (\verb+double+): 
\index{Header parameters!{\bf tsamp}}
time interval between samples (s)
\item {\bf nbits} (\verb+int+): 
\index{Header parameters!{\bf nbits}}
number of bits per time sample
\item {\bf nsamples} (\verb+int+): 
\index{Header parameters!{\bf nsamples}}
number of time samples in the data file (rarely used any more)
\item {\bf fch1}  (\verb+double+): 
\index{Header parameters!{\bf fch1}}
centre frequency (MHz) of first filterbank channel
\item {\bf foff}  (\verb+double+): 
\index{Header parameters!{\bf foff}}
filterbank channel bandwidth (MHz)
\item {\bf FREQUENCY\_START}  (\verb+character+): 
\index{Header parameters!{\bf FREQUENCY\_START}}
start of frequency table (see below for explanation)
\item {\bf fchannel}  (\verb+double+): 
\index{Header parameters!{\bf fchannel}}
frequency channel value (MHz)
\item {\bf FREQUENCY\_END}  (\verb+character+): 
\index{Header parameters!{\bf FREQUENCY\_END}}
end of frequency table (see below for explanation)
\item {\bf nchans} (\verb+int+): 
\index{Header parameters!{\bf nchans}}
number of filterbank channels
\item {\bf nifs} (\verb+int+): 
\index{Header parameters!{\bf nifs}}
number of seperate IF channels
\item {\bf refdm}  (\verb+double+): 
\index{Header parameters!{\bf refdm}}
reference dispersion measure (cm$^{-3}$ pc)
\item {\bf period}  (\verb+double+): 
\index{Header parameters!{\bf period}}
folding period (s)
\end{itemize}
A given header stream will contain most, but not necessarily all, of the 
above variables. 

In the general case, the data consists of {\bf nifs} polarization
channels of {\bf nchans} frequency channels of {\bf nbit} numbers. The
data stream following the header can then be thought of as 1-D array
of $N$ elements with indices running between 0 and $N-1$, where
\begin{displaymath}
	N = {\rm \bf nifs} \times {\rm \bf nchans} \times {\rm \bf nsamples},
\end{displaymath}
and {\bf nsamples} is the observation time divided by {\bf tsamp}.
Thus, for a given IF channel $i = (0,1,2,3)$ and frequency channel $c
= (0 \dots {\rm \bf nchans}-1)$, the array index for sample $s =
(0,1,2 \dots)$ is
\begin{displaymath}
s \times {\rm \bf nifs} \times {\rm \bf nchans}+ i \times {\rm \bf nchans} + c.
\end{displaymath}
The sky frequency of channel $c$ is then simply
\begin{displaymath}
		{\rm \bf fch1} + c \times {\rm \bf foff}.
\end{displaymath}
We follow the Parkes/Jodrell Bank
convention of assigning a negative frequency to {\bf
foff} in the headers to signify that the highest frequency channel is
{\bf fch1}.  Currently, all filterbank data is written out in this order
and the {\tt dedisperse} program relies on this fact in its dedispersing
algorithm (see \S \ref{reduction}).

Although this system works well for most applications, from version 2.3
there is a more flexible way of describing the frequency channels.
Instead of writing {\bf fch1} and {\bf foff}, it is now possible to 
write the individual frequency channel frequencies directly into the header
in the following way:
\begin{verbatim}
FREQUENCY_START f1 f2 f3 f4 FREQUENCY_END
\end{verbatim}
where \verb+f1+, \verb+f2+.... are the frequency channel
values in MHz. These may be in any order, {\em provided that}
the \verb+f1+ is the highest frequency (again this is 
stipulated because of {\tt dedisperse}'s algorithm).
This frequency table approach is used by the {\tt splice} program
to deal with non-contiguous data described next.

\clearpage
\section{Data conversion using {\tt filterbank} and {\tt splice}}
\label{filterbank}
\index{Programs!{\tt filterbank}}

The interface between the raw data and the rest of the SIGPROC package
is the {\tt filterbank} program. As with all the programs on-line help
is obtained by typing the name of the program followed by {\tt help}:
\input{filterbank.help}
Given just the name of the raw data file as the argument, {\tt
filterbank} will determine the origin of the data and, if it can read
the file, unpack the samples before writing the header parameters
and data as described in \S \ref{dataformat}. The header and data go
to the standard output by default but can be redirected to a file
using the {\tt -o filename} option, or in the standard way:
\begin{verbatim}
% filterbank rawdatafile > filterbankfile
\end{verbatim}
With no further options, {\tt filterbank} will read and unscramble all
the data in the original file. A specific portion of the data can be
specified using the {\tt -r} and {\tt -s} command-line options. For example:
\begin{verbatim}
% filterbank rawdatafile -r 10.0 > filterbankfile
\end{verbatim}
reads just the first 10 seconds of data. These options are useful for
a quick look at the data.

\subsection*{Selecting and/or summing IF streams}
\index{summing IFs}
\index{selecting IF streams}
By default, all the IF streams (if there are more than one) in the
file are read and processed. To select one or more of these, ignoring
the others, use the {\tt -i} option:
\begin{verbatim}
% filterbank rawdatafile -i 1 -i 2 > filterbankfile
\end{verbatim}
will process just the first two IF channels of the raw data file.
{\tt filterbank} provides the option to sum {\sl just the first two} IF
channels (to form total-power data) via the {\tt -sumifs} option:
\begin{verbatim}
% filterbank rawdatafile -sumifs > filterbankfile
\end{verbatim}
This is a useful, for example, to get just total power from
polarimetry data for off-line searching.

\subsection*{ASCII headers}
\index{ASCII headers}
As mentioned in \S \ref{dataformat}, {\tt filterbank} will broadcast a
header stream before writing the data. This header is used by other
downstream SIGPROC programs to process the data. To make use of it in
analysis with other programs, call the function \verb+read_header+ and
link with the other routines contained in the file
\verb+read_header.c+.  For those who prefer not to be bothered with
these routines, use the {\tt -headerfile} option when calling
filterbank. For example:
\begin{verbatim}
% filterbank B0823+26.pspm -headerfile > B0823+26.fil
\end{verbatim}
will create the file {\tt B0823+26.fil} containing just the filterbank
channels along with the relevant header parameters in an ASCII file
{\tt head}. In this case:
\begin{verbatim}
Original PSPM file: B0823+26.pspm
Sample time (us): 80.000002
Time stamp (MJD): 51740.882986111108
Number of samples/record: 512
Center freq (MHz): 430.000000
Channel band (kHz): 62.000000
Number of channels/record: 128
\end{verbatim}
the user is then left to parse this file as he/she feels fit.
An alternative means of getting header information would be
to use the {\tt header} program in the following example:
\begin{verbatim}
% filterbank B0823+26.pspm | header -tstart
\end{verbatim}
which will return {\tt 51740.882986111108} to the standard output.
Any of the header variable names listed in \S \ref{dataformat}
can be given as a command-line option to the {\tt header} program.
Further details are given in \S \ref{looking}.

\subsection*{Changing the number of bits per sample}
By default, {\tt filterbank} will write the outgoing data with the same
number of bits per sample as the native format (e.g.~4 bits per sample
for PSPM). For machines which write out larger numbers of bits
(e.g.~the WAPP) it is useful to be able to pack the data more
efficiently using the {\tt -n} option.  For example, the sequence:
\begin{verbatim}
% filterbank wappdatafile -n 8 > filterbankfile
\end{verbatim}
will process a WAPP data file (usually 16 bits per sample) and
pack the outgoing samples as single-byte integers. For search
purposes, where only marginal loss in sensitivity is seen and data products
are reduced significantly, use of this option is highly recommended.
For WAPP data, the loss in sensitivity from 16 to 8 bits is negligible,
packing down to 4 bits results in losses $\sim$5\%.

\subsection*{Floating-point output}
\index{floating-point output}
Currently, no descaling parameters are given in the header when
packing down data. This means that for applications where the
absolute value of the data is necessary (e.g.~polarization work)
it is necessary to store the data as floating-point numbers.
The option {\tt -floats} is provided for this purpose (although this
is really just an alias for {\tt -n 32}).

\subsection*{Byte swapping issues}
\index{byte swapping}
Multi-byte precision data are written in different orders depending on
your machine's operating system. The original WAPP data, for example,
was written on a PC (little endian format).  The {\tt filterbank}
program knows about this and {\sl automatically} does any byte
swapping required while reading. When it comes to writing the data
out, however, the program will always write data in the native order
of the processing machine.  To swap the bytes around before writing
for use on other machines, use the {\tt -swapout} option.

\subsection*{Correlator-specific options}
Presently, the WAPP is the only correlator machine recognized by SIGPROC
which records auto- and, in polarization mode, cross-correlation functions for 
given numbers of lags. The autocorrelation function $R(\tau)$, as a 
function of lag $\tau$ is defined by:
\begin{displaymath}
R(\tau) = \lim_{T\rightarrow\infty} \frac{1}{T} \int_0^T V(t) V^*(t+\tau) dt,
\end{displaymath}
where $V(t)$ is the complex signal voltage as a function of time $t$.
From the Weiner-Khinchin theorem, the power 
spectral density function $P(f)$ is the Fourier transform of $R(\tau)$:
\begin{displaymath}
P(f)=\frac{1}{2\pi} \int_{-\infty}^{+\infty} R(\tau) e^{-2\pi i f \tau} d\tau.
\end{displaymath}
In practice to obtain the equivalent of frequency channels of a
filterbank, the lags from each IF channel need to be corrected for
finite-level quantization --- the so-called van Vleck correction (see 
\index{van Vleck correction}
for example Hagen \& Farley 1973, Radio Science, {\bf 8}, 775--784)
before the Fast Fourier Transform (FFT) to obtain the spectra. For
reference, the three-level van Vleck formula used within {\tt filterbank} to 
correct measured auto-correlation values ($r$) to unbiased ones ($\rho$)
can be written as
\begin{displaymath}
r = \frac{1}{\pi} \int_0^{\rho} \left(
\exp \left( \frac{-(\alpha/\sigma)^2}{1+x} \right) +
\exp \left( \frac{-(\alpha/\sigma)^2}{1-x} \right) \right)
\frac{dx}{\sqrt(1-x^2)},
\end{displaymath}
where $\alpha$ is the digitizer threshold and $\sigma$ the rms
voltage. This correction is what {\tt filterbank} does by default
before FFTing the correlation functions to produce spectra. 

A number of options exist to modify the default processing.
To reduce FFT leakage, either a Hanning or Hamming window
\index{Hanning smoothing}
\index{Hamming smoothing}
can be applied to the correlation functions via the
{\tt -hamming} and {\tt -hanning} switches. Select {\tt -rawcfs} to
output the raw correlation functions quantized to the precision
specified by {\bf nbits}.  To get at the raw correlation functions,
include the floating-point option:
\begin{verbatim}
% filterbank wappdatafile -rawcfs -floats > rawcffile
\end{verbatim}
The {\tt -corcfs} option will write out the correlation 
functions {\sl applying} the van Vleck correction.

\subsection*{Obscure correlator options} For completeness, we mention two
other correlator specific options: {\tt -novanvleck} and {\tt
-zerolag}. The {\tt -novanvleck} option will not apply the
quantization correction before the FFT. This feature is really for
instructional purposes since, to FFT the data to get frequency
channels, signal-to-noise will be lost if the van Vleck correction
is not applied. Another option that
is primarily used for testing is {\tt -zerolag}. If selected, this
outputs just the first correlation function for each IF (the so-called
zero lag) as a floating-point number. Inserting $\tau=0$ into the
above expression for $P(f)$, we note that the zero lag is just the sum
over all the frequency channels --- equivalent to a time series with
no dispersion measure correction. 

For WAPP data, one final option is {\tt -invert}
\index{bandpass inversion}
which inverts the band after the FFT to change the frequency ordering.
This should normally be dealt with in the WAPP header but is included
here to process data where the header information about frequency
ordering is incorrect.

\subsection*{Splicing files}
\index{Programs!{\tt splice}}

Most data acquisition systems store the collected data as single
files per observation. For the new multiple WAPP system at Arecibo,
where each machine runs independently to sample a different part of
the band, a number of data files result for each frequency band.
In order to analyse these datasets together, the {\tt splice} program
will join multiple filterbank files, provided that they all have
on the same time stamp. The syntax is very simple:
\begin{verbatim}
	splice file1.fil file2.fil file3.fil > splice.fil
\end{verbatim}
where it is assumed that the input files \verb+file1.fil+, \verb+file2.fil+ and
\verb+file3.fil+ have already been converted into filterbank format
as described above. The resulting file, \verb+splice.fil+ in this
example, is also in filterbank format and can be read by subsequent
programs. Although the files need not span a contiguous
radio frequency band,
{\tt splice} will complain if the input files do not
all have the same time stamp, or if they are not ordered in
descending frequency order. The latter check is done so
that the data conform to the order expected by the 
dedispersion algorithm (\S \ref{reduction}).

\clearpage
\section{Creating mock data sets using {\tt fake}}
\index{Programs!{\tt fake}}

The {\tt fake} program was written to create test data sets containing
pulses hidden in Gaussian noise:
\label{fake}
\input{fake.help}
Default parameters are a filterbank similar to the PSPM. As an example, 
consider some fake PSPM data for a 42-s observation of a
pulsar with a period of $\sim\pi$ ms, a duty cycle of 10\% and a DM of 30:
\begin{verbatim}
% fake -period 3.1415927 -width 10 -dm 30 -tobs 42 -nbits 4 > pspm.fil
\end{verbatim}
Each channel of fake data has a zero mean and unit rms.  The
signal-to-noise ratio refers to the height of a single pulse in each
channel. In the above example, the default signal-to-noise was
used. Weaker pulsars can be easily made to challenge limits of
off-line search algorithms etc. By default, the fake pulse width $w$
is smeared by an amount dependent on the filterbank setup using the
quadrature sum:
\begin{displaymath}
	\sqrt{w^2 + {\rm \bf tsamp}^2 + t_{\rm DM}^2},
\end{displaymath}
where $t_{\rm DM}$ is the dispersion smearing of the pulse over a
single filterbank channel given by:
\begin{displaymath}
 t_{\rm DM} = 8.3 \times 10^6 {\rm ms} \, \, {\rm DM} \, \Delta \nu / \nu^3,
\end{displaymath}
assuming the centre frequency $\nu$ is much larger than the channel bandwidth
$\Delta \nu$ (both measured in MHz). Smearing can be disabled using the {\tt
-nosmear} option. Bit-format and byte-swapping options are identical
to those described for the {\tt filterbank} program in the previous
section. The starting seed of the random number generator defaults to
a number obtained by starting with the number of seconds since midnight
and calling the random number generator that many times. This can be 
overridden by specifying a seed using the {\tt -seed} option. 

\section{Looking at headers using {\tt header}}
\label{headers}
\index{Programs!{\tt header}}
The {\tt header} program allows humans easy access to the raw data
file, or the binary header string in the filterbank data format.  
As an example of the full
output, here is the header of our PSPM test data:
\begin{verbatim}
% header B0823+26.fil

Data file                        : B0823+26.fil
Header size (bytes)              : 191
Data size (bytes)                : 2359296
Data type                        : filterbank
Telescope                        : Arecibo
Datataking Machine               : PSPM
Frequency of channel 1 (MHz)     : 433.968000
Channel bandwidth      (MHz)     : -0.062000
Number of channels               : 128
Time stamp of first sample (MJD) : 51740.882986111108
Gregorian date (YYYY/MM/DD)      : 2000/07/15
Sample time (us)                 : 80.00000
Number of samples                : 36864
Observation length (seconds)     : 2.949120
Number of bits per sample        : 4
Number of IFs                    : 1
\end{verbatim}
alternatively, {\tt header} can be used with one or more of the
above command-line options to return just the value of the
parameter of interest (this is particularly useful when 
getting values from within scripts without having to parse
the standard output). Currently available options are:
\begin{verbatim}
-telescope  - return telescope name
-machine    - return datataking machine name
-fch1       - return frequency of channel 1 in MHz
-foff       - return channel bandwidth in MHz
-nchans     - return number of channels
-tstart     - return time stamp of first sample (MJD)
-tsamp      - return sample time (us)
-nbits      - return number of bits per sample
-nifs       - return number of IF channels
-headersize - return header size in bytes
-datasize   - return data size in bytes if known
-nsamples   - return number of samples if known
-tobs       - return length of observation if known (s)
\end{verbatim}
It should be noted that {\bf headersize}, {\bf datasize}, {\bf
nsamples} and {\bf tobs} are not header variables {\it per se};
they are derived by the program, based upon the file size and the real
header variables.

\clearpage
\section{Looking at data using {\tt bandpass}, {\tt reader} and {\tt pgplotter}}
\index{Programs!{\tt bandpass}}
\label{looking}
The {\tt bandpass} program is a simple utility to read incoming
data and output a time-averaged bandpass:
\input{bandpass.help}
In its simplest form, {\tt bandpass} averages over the entire
data file. The data for Fig.~\ref{0823band} were obtained using:
\begin{verbatim}
% filterbank B0823+26.pspm | bandpass > bandpass.ascii
\end{verbatim}
\begin{figure}[hbt]
\setlength{\unitlength}{1in}
\begin{picture}(0,2.5)
\put(1.2,3.2){\special{psfile=0823band.ps hscale=40 vscale=40 angle=270}}
\end{picture}
\caption{\sl Output data from {\tt bandpass} for the test
observation of PSR B0823+26 using the PSPM.}
\label{0823band}
\end{figure}
The ASCII data is written in a simple format with one line
for each frequency channel: \verb+frequency if1 if2...+ for
up to {\bf nifs} seperate IFs. The {\tt -d} and {\tt -t} 
options allow averaging and output of the bandpass for a
given number of dumps, or seconds. Each dump is encapsulated
within \verb+#START+ and \verb+#STOP+ separators:
\begin{verbatim}
#START
freq(1)      if(1) .... if(nifs)
 ...           .    ..    ...
freq(nchans) if(1) .... if(nifs)
#STOP
\end{verbatim}
where the \verb+freq(1)+ is the sky frequency of channel 1 in MHz and
so on for all {\bf nchans} channels. Although plotting is left up to
the users discretion in general, SIGPROC provides a little PGPLOT
utility {\tt pgplotter} which plots data streams passed in this
format. For example, try
\begin{verbatim}
% filterbank B0823+26.pspm | bandpass | pgplotter
\end{verbatim}
\index{Programs!{\tt pgplotter}}

Another useful program is {\tt reader} which will
print out filterbank-format data as an ASCII stream to the 
standard output. 
\input{reader.help}
In the general case, a filterbank file with 
{\bf nchans} channels and {\bf nifs} IFs, output is of the form:
\begin{verbatim}
% reader filterbankfile

time(1) if(1)c(1) if(1)c(2) .... if(1)c(nchans) ...... if(nifs)c(nchans)
time(2) if(1)c(1) if(1)c(2) .... if(1)c(nchans) ...... if(nifs)c(nchans)
time(3) if(1)c(1) if(1)c(2) .... if(1)c(nchans) ...... if(nifs)c(nchans)
\end{verbatim}
the default case is to print out all IF and frequency channels.
The output can be tailored by the {\tt -i} and {\tt -c} options
to get just specific channels of interest. For example:
\begin{verbatim}
% filterbank B0823+26.pspm | reader -c 1 -c 2 -c 3 -c 4 | head

0.000000 5.000000 5.000000 6.000000 5.000000 
0.000080 7.000000 5.000000 7.000000 6.000000 
0.000160 7.000000 5.000000 6.000000 6.000000 
0.000240 5.000000 5.000000 5.000000 7.000000 
0.000320 5.000000 4.000000 5.000000 6.000000 
0.000400 5.000000 4.000000 5.000000 6.000000 
0.000480 5.000000 5.000000 5.000000 6.000000 
0.000560 5.000000 5.000000 6.000000 5.000000 
0.000640 6.000000 4.000000 5.000000 7.000000 
0.000720 6.000000 6.000000 6.000000 5.000000 
\end{verbatim}
shows just the first four frequency channels of the PSPM data
as a function of time. The {\tt -numerate} switch will
change this time stamp to an integer counter. Time or
integer counters can be turned off completely via the
{\tt -noindex} option. The {\tt -stream} option will,
\index{Data formats!{\tt -stream}}
as in the case of the continuous {\tt bandpass} output
above, output a stream of numbers encapsulated by
\verb+#START+ and \verb+#STOP+ separators. As before, this
format may be passed to {\tt pgplotter} for plotting.
\index{Programs!{\tt reader}}

\section{Data reduction using {\tt decimate} and {\tt dedisperse}}
\label{reduction}
\index{Programs!{\tt decimate}}

Adding of adjacent time and/or frequency channels 
together to reduce the original resolution and size of the original
data file is possible using the {\tt decimate} program:
\input{decimate.help}
Output data from {\tt decimate} is in standard filterbank format
so that it can be easily read in by other SIGPROC programs.
To get ASCII data, use the {\tt reader} program (see \S \ref{looking}).
The following example adds all the frequency channels together,
and every 32 time samples, to create the time series shown in
Fig.~\ref{0823time}.
\begin{verbatim}
% filterbank B0823+26.pspm | decimate -t 32 -n 32 | reader > timeseries.ascii
\end{verbatim}
\begin{figure}[hbt]
\setlength{\unitlength}{1in}
\begin{picture}(0,2.5)
\put(1.2,3.2){\special{psfile=0823time.ps hscale=40 vscale=40 angle=270}}
\end{picture}
\caption{\sl Output time series from {\tt decimate} for the test
observation of PSR B0823+26 using the PSPM.}
\label{0823time}
\end{figure}
Note that we have used the {\tt -n} option to force the output
number of bits per sample to be 32. By default {\tt decimate}
outputs data with the same number of bits as the incoming filterbank
data. In this case, where there are strong single pulses,
adding all the channels together would result in a signal-to-noise
loss when trying to write the output time series with 4-bit precision.

While {\tt decimate} is a good means for getting time series of weakly
dispersed pulsars, it does not take into account the effects of
dispersion by the interstellar medium where pulses emitted at higher
radio frequencies travel faster through the interstellar medium,
arriving earlier than those emitted at lower frequencies.  The time
delay $\Delta t$ between a high frequency $\nu_{\rm hi}$ relative to a
lower on $\nu_{\rm lo}$ is \index{dedispersion}
\begin{displaymath}
 \Delta t = 4.15 \times 10^6 \, \, {\rm ms} \, \,
 \times (\nu_{\rm lo}^{-2} - \nu_{\rm hi}^{-2})  
 \times {\rm DM},
\end{displaymath}
where the frequencies are in MHz and the dispersion measure 
${\rm DM} = \int_{\rm 0}^{d} \,\, n_{\rm e} \,\, dl$
(cm$^{-3}$ pc) is the integrated
column density of free electrons along the line of sight.
Here, $d$ is the distance to the pulsar (pc) and $n_{\rm e}$ is the
free electron density (cm$^{-3}$). For distant high-DM pulsars,
especially those with short periods, dispersion needs
to be accounted for to retain full time resolution.  The {\tt
dedisperse} program does this by adding frequency channels with the
appropriate time delays given a DM value:
\input{dedisperse.help}
The dedispersion algorithm reads in a block of data and
gets the appropriately delayed sample by looking forward in
the array. This there requires that the frequency channels
are passed down in descending frequency order and {\tt dedisperse}
will complain if this condition is not met!
\index{Programs!{\tt dedisperse}}

In the example data for the 1.5578-ms pulsar B1937+21 shown in 
Fig.~\ref{1937giant}, the left panel was produced via:
\begin{verbatim}
% filterbank B1937+21.539 | dedisperse -d 71.04 | reader > timeseries.ascii
\end{verbatim}
\begin{figure}[hbt]
\setlength{\unitlength}{1in}
\begin{picture}(0,2)
\put(-0.1,2.4){\special{psfile=1937whole.ps hscale=33 vscale=33 angle=270}}
\put(+3.7,2.2){\special{psfile=1937bands.ps hscale=28 vscale=28 angle=270}}
\end{picture}
\caption{\sl A WAPP observation of the millisecond pulsar B1937+21
showing a single ``giant''pulse. Left: the dedispersed time series over the
entire 100-MHz band. Right: the pulse seen in four dedispersed 25-MHz 
subbands. The length of the time series segment is $\sim0.26$ s. 
The sampling time is 63.32 $\mu$s.}
\label{1937giant}
\end{figure}
This single pulse is shown in four dedispersed frequency subbands in
the right-hand panel of Fig.~\ref{1937giant}. These were obtained
by adding a {\tt -b 4} option into the dedisperse
command-line in the above pipeline. In this case, dedispersion is
carried out relative to the frequency of the first summed channel in
each of the bands.

\section{Getting pulse profiles using {\tt fold} and {\tt profile}}
\label{folding}
\index{Programs!{\tt fold}}

Obtaining integrated pulse profiles, and single pulses, from your
data files is possible using the {\tt fold} program which allows you
to fold filterbank data modulo a pulse period to produce
pulse profiles. In addition, there is now a basic ASCII viewing
program {\tt profile} which displays profiles from {\tt fold} to the
standard output. {\tt fold} accepts
any number of IF and/or frequency channels, producing
{\bf nifs} $\times$ {\bf nchans} sets of profiles. The folding
algorithm used is a simple one: for each time sample, compute
the phase based on a, possibly time-dependent, value of the
pulse period and add that sample to the nearest phase bin of
the appropriate profile. The synopsis of {\tt fold} is summarized below:
\input{fold.help}

\subsection*{Folding data at a fixed period} Consider folding a series containing
our fake $\sim\pi$-ms pulsar:
\begin{verbatim}
% fake -period 3.14159 -nchans 1 -nbits 32 | fold -p 3.14159 > profile.ascii
\end{verbatim}
Note that the default profile output is in ASCII format. 
\index{Data formats!{\tt -ascii}}
This may be substituted by EPN or PSRFITS using the {\tt -epn} or
{\tt -psrfits} options on the command line. The
format of this output is a line for each bin:
\begin{verbatim}
bin_number if(1)c(1) if(1)c(2) .... if(1)c(nchans) ...... if(nifs)c(nchans)
\end{verbatim}
In order to avoid overflows during folding, {\tt fold} will by default
subtract an offset from each folded sample calculated as the median
value of a given data block. To turn off this feature, use the {\tt
-nobaseline} option.  The default number of bins is given by the next
largest integer value to the ratio of the folding period divided by
the sampling time. This is, however, completely flexible. A lower
number of bins would be desirable, for example, when folding data for
a faint pulsar or candidate.  {\tt fold} will permit oversampling
\index{oversampling}
which can pay dividends for high signal-to-noise observations of
short-period pulsars.

A useful feature of {\tt fold} for weak pulsars, and those for which the 
pulse happens to lie on the edge of the window is the {\tt -m} option
which allows the display of multiple pulses. For example, try:
\begin{verbatim}
% fake -period 3.14159 -nchans 1 -nbits 32 | fold -p 3.14159 -m 2 | pgplotter
\end{verbatim}

\subsection*{Folding data using polynomial coefficients} For practical
applications, the apparent pulse period is time-variable during
the integration due to Doppler shifts resulting from the Earth's
motion and (for binary pulsars) from Doppler shifts induced by
orbiting companions. To account for these the folding period 
needs to be updated during the integration. The {\sc TEMPO}
\index{Software Packages!{\sc TEMPO}}
timing package can be used to create a set of polynomial coefficients
to predict the change in period with time and {\tt fold} can
read these ``polyco'' files from {\sc TEMPO} for these
purposes. A script to run {\sc TEMPO} to produce these files
is described in \S \ref{polyco}.
To tell {\tt fold} to read a polyco file, supply
the name of the filename with the {\tt -p} option.
\begin{verbatim}
% filterbank B0823+26.pspm | fold -p polyco.dat -n 128 -epn > B0823+26.epn
\end{verbatim}
will fold each channel of the sample PSPM data for PSR B0823+26 to
produce 128-bin profiles written to the file {\tt B0823+26.epn} in
EPN format. If no {\tt -p} option is given to {\tt fold} the program
will look for the file {\tt polyco.dat} as a matter of course so
that, in the above case, it was not strictly necessary to specify
the name of the polyco file. This is assumed in the following pipeline
where the data are first dedispersed at the reference DM value of
19.4 cm$^{-3}$ pc before being passed to {\tt fold}:
\begin{verbatim}
% filterbank B0823+26.pspm | dedisperse -d 19.4 -epn | fold > B0823+26.epn
\end{verbatim}
\index{polyco.dat}

\subsection*{Getting sub-integrations} In the above examples,
{\tt fold} produces one profile for each of {\bf nchans} $\times$ {\bf nifs}
incoming data streams which corresponds to folding over the entire data 
set. It is often desirable to look at sub-profiles dumped at regular
intervals during the observation --- the {\tt -d} (dump) option allows you 
to do this. Specifying a floating-point number, say $f$ seconds, in this 
mode will output the accumulated profile every $f$ seconds.
The following example on
our fake millisecond pulsar data would dump a subintegration exactly
every 15 seconds:
\begin{verbatim}
% fold fakepulsar.fil -d 15.0 -p 3.1415927 -epn > fakeprofiles.epn
\end{verbatim}
Supplying an integer argument with the {\tt -d} option, 
say $n$, the profiles are dumped every $n$ pulses. So {\tt -d 15}
in the above example
results in a profile being dumped every 15 periods (about 47 ms).

\subsection*{Single pulses and windowing profiles} Individual pulses
\index{single pulses}
can be obtained by specifying {\tt -d 1} to the {\tt fold}
command line. The following example demonstrates this for the
PSR B0823+26 PSPM data:
\begin{verbatim}
% filterbank B0823+26.pspm | dedisperse -d 19.4 | fold -d 1 -epn > B0823+26.epn
\end{verbatim}
The resulting EPN file contains a record for each single pulse.
For this short data set, this amounts to just
five single pulses shown in Fig.~\ref{0823sps}.
\begin{figure}[hbt]
\setlength{\unitlength}{1in}
\begin{picture}(0,1.5)
\put(-0.25,2){\special{psfile=nowindow.ps hscale=67 vscale=67 angle=270}}
\put(-0.25,1){\special{psfile=window.ps   hscale=67 vscale=67 angle=270}}
\end{picture}
\caption{\sl Top: dedispersed single pulses for
the PSPM test observation of PSR B0823+26. Bottom: the same data
set after applying a phase window of 0.825 to 0.925 (see text)}
\label{0823sps}
\end{figure}

For single-pulse applications, where the off-pulse region of the
\index{pulse windowing}
profile is usually not interesting, it is desirable to be able to set
a window around the pulse. The {\tt fold} program allows setting of
windows via the {\tt -l} and/or {\tt -r} command-line options which
specify the left and right-hand phase values of the windows. 
Phase values should be specified in turns ranging between 0.0
and 1.0. For example, the pulses in the lower panel of Fig.~\ref{0823sps}
were obtained using {\tt fold -l 0.825 -r 0.925} for the PSR B0823+26
dataset. As before for the full profile, unless specified otherwise,
{\tt fold} will choose the number of bins based on the size of
the window divided by the sampling interval. 

All of the above examples have used a seperate plotting program
to produce the profiles for the figures. Since each user tends
to have his/her favourite method for producing such plots, no
facility exists within SIGPROC to to this. To get a quick look
at profiles, there is now a program {\tt profile} which will
display ASCII representations to the standard output. The program
has two modes of operation: 2-D profile ``plots'' or 1-D grey-scale
representations. To get a 2-D profile - the output from fold needs
to come in the standard ASCII format. For example, let's create
and fold data from a 1-s pulsar:
\begin{verbatim}
fake -period 1000.0 -nchans 1 | fold -p 1000.0 | profile
\end{verbatim}
The output from {\tt profile} would then be a mock 2-D profile
and will look something like this:
\begin{verbatim}
                               ##                               
                               ##                               
                               ##                               
                               ##                               
                               ##                               
                               ##                               
                               ##                               
                               ##                               
                               ##                               
                               ##                               
                               ##                               
                               ##                               
                               ###                              
                              ####                              
                              ####                              
                              ####                              
   #                          ####                              
####   #  # ##  #    #   # ## ##### #  ##  # # #     #### ### ##
################################################################
\end{verbatim}

To get a 1-D pseudo ``grey-scale'' plot, the profiles need
to be output from fold using the {\tt stream} option. This is
particularly useful if you wish to display profiles as sub-integrations
or folded frequency channels. For example, to display additions of
every 2 pulses from our 1-s fake pulsar we would do the following:
\begin{verbatim}
fake -period 1000.0 -nchans 1 | fold -p 1000.0 -stream -d 2 -nobaseline| profile
\end{verbatim}
to produce the following output showing subintegration index and
elapsed time to the left of each profile:
\begin{verbatim}
|0001|00:00:00|                              ,$#,                              |
|0002|00:00:02|                               $#,                              |
|0003|00:00:04|                              ,#$,                              |
|0004|00:00:07|                              ,@#:                              |
|0005|00:00:09|                               ##,                              |
|0006|00:00:09|                               $#,                              |
\end{verbatim}
Where the {\tt -nobaseline} option during folding has been used to 
preserve the original levels which would otherwise get changed each
subintegration. Note also that the last ``subintegration'' in this
plot is in fact the integrated profile which is dumped by {\tt fold}
as a matter of course. Frequency channel plots can be produced in a 
similar way. For example, try creating a 100-ms pulsar with 32 channels
and a DM of 500 and piping this through {\tt fold} and {\tt profile}.
\begin{verbatim}
fake ..... | fold -p 100.0 -stream -nobaseline | profile -frequency
\end{verbatim}
where the {\tt -frequency} option will be required to label
the frequency channels correctly.

\section{Putting it all together: the {\tt quicklook} data reduction script}
\label{quicklook}
\index{Programs!{\tt quicklook}}
As an application of most of the programs discussed in the preceding
sections, we conclude with the {\tt quicklook} data analysis script
which is designed to dedisperse and fold raw pulsar machine data taken
on a known pulsar and produce a diagnostic output plot summarizing
various aspects of the data:
\input{quicklook.help}
%Most of the above command-line options should be obvious given
%the above explanation and the discussions of various aspects
%of the preceding programs. 
%{\tt quicklook} will try to construct
%a source name based on the name of the raw datafile and generate
%a polyco file. If it fails to do this, use the {\tt -psr} option
%to specify the name of the pulsar required. The {\tt -mypolyco}
%option will use a pre-existing {\tt polyco.dat} file to calculate
%the period and get the dispersion measure. Alternatively it is
%possible to process the data using a fixed period and dispersion
%measure with the {\tt -period} and {\tt -dm} options. 
As an example, the command:
\begin{verbatim}
% quicklook J1713+0747.744 -nbands 64 -nsints 32 -read 5
\end{verbatim}
reduces the first 5 seconds of a WAPP observation of the 
millisecond pulsar J1713+0747 producing the plot shown in Fig.~\ref{1713}.
N.B.~Use of this script assumes that you have the {\tt quickplot}
program compiled (see \S \ref{install} for further details).

\clearpage
\begin{figure}[hbt]
\setlength{\unitlength}{1in}
\begin{picture}(0,6.8)
\put(-0.1,-0.6){\special{psfile=J1713+0747.744.ps hscale=80 vscale=80}}
\end{picture}
\label{1713}
\caption{\sl Sample output from the {\tt quicklook} script for the
millisecond pulsar J1713+0747 observed by the WAPP. Top panel shows a
1024-point average of the dedispersed time series which has been
normalised (as far as possible) so that it has a zero mean and unit
rms. Below this are two panels showing the dedispersed frequency
sub-bands (which clearly show the dispersion of the pulsar) and time
sub-integrations as a function of pulse phase. The bottom plot
is the integrated pulse profile. The signal-to-noise rato of this
profile is reported at the top of the plot along with essential header
information.}
\end{figure}
\clearpage

\section{Version history and plans for future work}
\label{past/future}
The file {\tt version.history} summarizes the work done
on SIGPROC to date:
\input{version.history}

\noindent
All suggestions for improvements, including (best of all!)
offers to contribute write and/or improved routines for future 
releases of the package are most welcome via email:
{\tt drl@jb.man.ac.uk}.

\section*{Acknowledgements}

In putting together the SIGPROC package, I had the good fortune to
work with a number of people who kindly donated existing routines or
offered to write new ones.  Andy Dowd wrote the original version of
what became the {\tt wapp2fb} routine for converting raw WAPP
correlation values into spectra. Jeff Hagen also contributed to this
effort and wrote the routines used for reading WAPP headers and byte
swapping. Ingrid Stairs provided the \verb+pspm_decode+ routine --- a
C-version of an original Fortran--77 subroutine written by Alex
Wolsczcan. Ingrid also debugged {\tt dedisperse} so that it can handle
WAPP timing-mode data.  Mike Keith and Ralph Eatough added the binary
options to fake. Use was also made of some Numerical Recipes
routines for FFTs and random number generation in {\tt fake}. Finally,
many thanks to Jim Cordes, Maura Mclaughlin, Ramesh Bhat, Ingrid
Stairs and Joanna Rankin
for their help in putting together and debugging some of the
routines, and their suggestions for functionality.

\clearpage
\appendix
\section{Monitoring programs using the Tk {\tt monitor} widget}
\label{monitoring}
\index{Software Packages!Tcl/Tk}
\index{monitoring programs}
The SIGPROC programs run without any messages to the standard
output. To keep track of their progress, a {\tt wish} script:
{\tt monitor} can run in the background. 

To start the {\tt monitor} script running, go to the directory
where you are processing your files and type {\tt monitor}.
For example, starting a filterbank command:
\begin{verbatim}
% filterbank J1713+0747.744 -sumifs > J1713+0747.744.fil
\end{verbatim}
will cause the following status bar to appear in the upper left-hand
corner of the screen:

\begin{figure}[hbt]
\setlength{\unitlength}{1in}
\begin{picture}(0,0.5)
\put(0.4,-3.6){\special{psfile=monitor.ps hscale=70 vscale=70}}
\end{picture}
\end{figure}

This counter will tick away updating as the file gets updated
until the program is finished. If you have several jobs running,
in a pipeline for example, several status bars will appear
until their respective job is completed. To stop the monitor
script at any time, type:
\begin{verbatim}
% monitor off
\end{verbatim}
Once the program
is finished what it is doing, the monitor will go to sleep
and wait for another SIGPROC program to start.

Note that you will confuse the script if you have two
programs running from the same directory (for example
two {\tt filterbank} processes running on different
raw datafiles) since the {\tt programname.monitor} file
will get updated by both programs. For such applications,
run the programs from separate directories.

The {\tt monitor} script polls programs by looking at 
logfiles which are written whenever the file {\tt monitor.running}
exists in the working directory. If you prefer not to
use {\tt monitor} but would like to look at these logfiles,
simply create the file {\tt monitor.running} in your area
\begin{verbatim}
% touch monitor.running
\end{verbatim}
then run a SIGPROC program and {\tt tail} the resulting
{\tt programname.monitor} file to keep track of what is
going on. For example, while {\tt dedisperse} is running
in a pipeline you would see the following:
\begin{verbatim}
% tail dedisperse.monitor
input stdin status time:0.1s:DM:35.0pc/cc output stdout
input stdin status time:0.2s:DM:35.0pc/cc output stdout
\end{verbatim}
To turn off this logging mode, simply delete the 
{\tt monitor.running} file.

\section{Running {\sc TEMPO} to generate polynomial coefficients}
\label{polyco}
\index{Software Packages!{\sc TEMPO}}
\index{polyco.dat}
If you have {\tt expect} in your path, you will also be able
to take advantage of {\tt polyco} a simple script designed
to take the pain out of running {\sc TEMPO} to generate 
files containing polynomial coefficients for use by {\tt fold}.
The synopsis of {\tt polyco} is as follows:
\input{polyco.help}
It is assumed that you have {\sc TEMPO} installed on your
computer so that the {\tt tempo} executable file is in
your path, and the {\tt TEMPO} environment variable set.
At Arecibo, a solaris version of {\sc TEMPO} can be found
in {\tt /home/pulsar/bin/tempo} and the {\tt TEMPO} 
environment variable should be set to {\tt /home/pulsar/tempo11}.

Running {\tt polyco} is then a matter of giving a pulsar name
from the list of ephemeredes contained in \verb+$TEMPO/tztot.dat+
and the start and stop MJD ranges over which you wish the 
coefficients to apply. The default is to generate coefficients
for use at the time you run {\sc TEMPO}.

\section{The EPN data format}
\label{epn}
\index{Data formats!EPN}
The {\bf E}uropean {\bf P}ulsar {\bf N}etwork (``{\bf EPN}'') is an
association of European astrophysical research institutes that
co--operate in the subject of pulsar research. The EPN format
was developed for the exchange of pulse profiles between different
groups of individuals to permit a free interchange of data.
The following text was taken from a paper which originally
appeared in Astronomy \& Astrophysics Supplement Series  (1998)
{\bf 128} 541--544 and is included here for quick reference.

Each EPN file consists of one or more blocks.  The basic structure
of an EPN block is shown in Fig.~\ref{epnblock}.  
Each file has a common fixed
length {\it header} followed by a number of individual {\it data
streams} of equal length. The header describes the data, containing
information on the pulsar itself, the observing system used to make
the observation as well as some free-form information about the
processing history of the data. The onus is on the site--specific
conversion process to ensure correct conversion to the standardized
entries and reference to common catalogues (e.g.~the Taylor et al.~1993
catalogue of pulsar
parameters).  The full list of header variables is given in Tables 
\ref{epnheader} and \ref{epnsubheader}.

\begin{figure}[hbt]
\small
\begin{center}
\begin{minipage}{3.8cm}
\fbox{
 \parbox{3.5cm}{\begin{center} {\bf  Header } \\  480 Characters \end{center}} 
      }
\fbox{
 \begin{minipage}{3.5cm}
  \fbox{ \parbox{3.1cm}{\begin{center} {\bf Sub-Header } \\ 160 Characters \end{center} }} 
  \fbox{ \parbox{2.96cm}{\begin{center}  {\bf Data } \\   \end{center} } }
 \end{minipage}
       }
\fbox{
 \begin{minipage}{3.5cm}
  \fbox{ \parbox{3.1cm}{\begin{center} {\bf Sub-Header } \\ 160 Characters \end{center} }} 
  \fbox{ \parbox{2.96cm}{\begin{center}  {\bf Data } \\   \end{center} } }
 \end{minipage}
       }
\fbox{
 \begin{minipage}{3.5cm}
  \fbox{ \parbox{3.1cm}{\begin{center} {\bf Sub-Header } \\ 160 Characters \end{center} }} 
  \fbox{ \parbox{2.96cm}{\begin{center}  {\bf Data } \\  \end{center} } }
 \end{minipage}
       } 
\fbox{
 \begin{minipage}{3.5cm}
  \fbox{ \parbox{3.1cm}{\begin{center} {\bf etc. ...}  \end{center} }} 
 \end{minipage}
} 
\end{minipage}
\end{center}
\caption{\sl Schematic representation of an EPN data block.}
\label{epnblock}
\end{figure}

The data streams themselves may be outputs of different
polarization channels, or individual channels (bands) of a filterbank
or a combination thereof. In total, there may be $N_{\rm freq}$ data
streams of i.e. different frequencies for each polarization.  Each
data stream starts with a small, fixed length sub-header in front of
the actual data values.  The number of data streams and their length may
vary between different EPN files, but is constant within each file.  A
character field and an ordinal number is provided for each stream for
its identification. 
 
\begin{table}
\begin{center}
\footnotesize
\begin{tabular}{|rcccp{8cm}|}
\hline
Position & Name & Format & Unit & Comment \\
\hline 
\hline
1   &  version &  A8 & &  EPN + version of format (presently EPN05.00)\\
9   &  counter &  I4 & &  No. of records contained in this data block\\
13  &  history & A68  &  &comments and history of the data \\
\hline
81    & jname &  A12 & &  pulsar jname \\
93   & name & A12 &  & common name              \\
105   & $P_{\rm bar}$ &  F16.12 & s & current barycentric period\\
121    & DM      &  F8.3   & pc cm$^{-3}$& dispersion measure\\
129   & RM      &  F10.3  & rad m$^{-2}$ & rotation measure \\
139  & CATREF &  A6   & & pulsar parameter catalogue in use \\
145   & BIBREF    &  A8     & & bibliographical reference key (or observer's name) \\
153   &         & 8X   &  &  blank space free for future expansion \\
\hline
161   & $\alpha_{2000}$  & I2,I2,F6.3 & hhmmss& right ascension of source \\
171   & $\delta_{2000}$ & I3,I2,F6.3 & ddmmss& declination of source\\
182  & telname   & A8  & & name of the observing telescope (site) \\
190  & EPOCH  & F10.3 & day & modified Julian date of observation \\
200  & OPOS   & F8.3  & degrees & position angle of telescope \\
208  & PAFLAG & A1    &  &  A = absolute polarization position angle, else undefined\\
209  & TIMFLAG & A1   &  &  A = absolute time stamps (UTC), else undefined \\
210 &         & 31X   &  &  blank space free for future expansion \\
\hline
241  & $x_{\rm tel}$& F17.5& m & topocentric X rectangular position of telescope \\
258  & $y_{\rm tel}$& F17.5& m & topocentric Y rectangular position of telescope \\
275  & $z_{\rm tel}$& F17.5& m & topocentric Z rectangular position of telescope \\
292 &         & 29X   &  &  blank space free for future expansion \\
\hline
321  & CDATE & I2,I2,I4 & d m y & creation/modification date of the dataset \\
329  & SCANNO & I4 &  & sequence number of the observation \\
333  & SUBSCAN & I4 &  &sub--sequence number of the observation \\
337  & $N_{\rm pol}$ & I2 &  & number of polarizations observed \\ 
339  & $N_{\rm freq}$ & I4 & & number of frequency bands per polarisation \\
343  & $N_{\rm bin}$ & I4  & & number of phase bins per frequency (1-9999) \\
347  & $t_{\rm bin}$ & F12.6 & $\mu$s & duration (sampling interval) of a phase bin \\
359  & $t_{\rm res}$ & F12.6 & $\mu$s & temporal resolution of the data \\
371  & $N_{\rm int}$ & I6  & & number of integrated pulses per block of data \\
377  & $n_{\rm cal}$ & I4 & $t_{\rm bin}$ & bin number for start of calibration signal\\
381  & $l_{\rm cal}$ & I4 & $t_{\rm bin}$ & length of calibration signal \\
385  & FLUXFLAG      & A1 &    & F = data are flux calibrated in mJy, else undefined \\
386  & & 15X  & & blank space free for future expansion \\
\hline
401 &         & 80X   &  &  blank space free for future expansion \\
\hline
\end{tabular}
\normalsize
\caption{\sl A description of the EPN format variables.}
\label{epnheader}
\end{center}
\end{table}

\begin{table}
\begin{center}
\footnotesize
\begin{tabular}{|rcccp{8cm}|}
\hline
Position & Name & Format & Unit & Comment \\
\hline 
\hline
481  & IDfield &  A8 & &  type of data stream (I,Q,U,V etc.) \\
489  & $n_{\rm band}$& I4 & &  ordinal number of current stream  \\
493  & $n_{\rm avg}$ & I4 & & number of streams averaged into the current one \\
497  & $ f_0$ & F12.8 &  & effective centre sky frequency of this stream\\
509  & $ U_f$ & A8 &  & unit of $f_0$ \\
517  & $ \Delta f $&  F12.6 &  & effective band width \\
529  & $ U_{\Delta} $&  A8 &  & unit of $\Delta f$\\
537  & $ t_{\rm start} $ & F17.5 & $\mu$s& time of first phase bin w.r.t. EPOCH \\
554  &               & 7X    &        & blank space free for future expansion\\
\hline
561  & SCALE  & E12.6 &  & scale factor for the data\\
573  & OFFSET & E12.6 &  & offset to be added to the data \\
585  & RMS    & E12.6 &  & rms for this data stream\\
597   & $P_{\rm app}$ &  F16.12 & s & apparent period at time of first phase bin\\
613  &     &  28X    &       & blank space free for future expansion\\
\hline
641 & Data(1)& I4 & & scaled data for first bin \\
$ 4 (N_{\rm bin}-1)+641$& Data($N_{\rm bin}$) & I4 &  & 
data for last bin of stream\\
\hline 
$640 + N_{\rm records}*80$& & & & end of first stream \\
\hline
\end{tabular}
\normalsize
\caption{\sl The sub-header variables within an EPN file}
\label{epnsubheader}
\end{center}
\end{table}

\subsection*{Format Compatible Software}

To incorporate the capability to read and write data in this format
within existing analysis software, a simple routine exists which can
read and write data in this format. In addition, we have written some
sample programs which can plot the data and display the header
parameters.  The software are written in 
{\it Fortran---77}\footnote{some simple C utilities are planned for
a future version of SIGPROC} and have
been packaged into a single UNIX tar file which is freely available
via the {\it Internet} from 
\verb+http://www.jb.man.ac.uk/~drl/download/epn.tar.gz+

To uncompress and extract the contents of the tar file on a UNIX
operating system, issue the commands:

\begin{verbatim}
% gunzip epnsoft.tar.gz
% tar xvf epnsoft.tar
\end{verbatim}

\noindent
The present package contains some sample data and two example programs ---
{\tt plotepn} and {\tt viewepn} which plot and view EPN files respectively.
The ASCII file {\bf 00README} in this packages gives further details of the 
software and how to use it.

\section{Barycentre correction using {\tt barycentre}}

A simple utility \verb+barycentre+ exists to convert a time
series or filterbank file to an equivalent frame which is
at rest with respect to the solar system barycentre.
The synposis of the program is
\input{barycentre.help}

By default, {\tt barycentre} runs {\tt TEMPO} to create a file
{\tt polyco.bar} which it then uses to perform the correction.
If you would prefer to supply a file to do this, create one in
advance, call it {\tt polyco.bar} and use the {\tt -mypolyco} option. 

%\section{Searching for pulsars using SEEK}
%
%The SEEK package searches for radio pulsars in noisy data sets.  So
%far SEEK has found over 30 pulsars in a
%number of projects with different telescopes and data acquisition
%systems.  Since SEEK now operates on SIGPROC formatted data, the
%entire package will shortly be merged into a future SIGPROC
%release. Full documentation on this package will then be available
%within this manual. The key programs are:
%
%\bigskip
%\noindent {\tt find} - a program to look for periodic signals in noisy
%time series.
%
%\smallskip
%\noindent {\tt best} - report the best candidates from \verb+find+'s 
%analysis
%
%\smallskip
%\noindent {\tt hunt/accn} - scripts to search data 
%using \verb+find+ over a range of DMs and trial accelerations.

%\bigskip
%\noindent
%SEEK is currently available on the web at
%\verb+http://www.jb.man.ac.uk/~drl/seek+.

\section*{}
\printindex
\addcontentsline{toc}{section}{Index}

\end{document}
