%%%%%%%%%%%%%%%%%%%%%%%%%%%%%%%%%
%%%Template for documenting CASPER Library Blocks%%%
%%%%%%%%%%%%%%%%%%%%%%%%%%%%%%%%%

%%%\Block describes a Block. The simulink name should be exactly as it appears in 
%%%Simulink, note that \_ must be used for underscores.
%%%The block label is the same as the simulink name but without the underscores.
%%%This label is defined as a latex reference and is
%%%usefull for cross-referencing other blocks. You can include in your
%%%Block Descriptions commands like refer to the adder tree block \ref{addertree}.
%%%The Block description can include arbitrary Latex, but do not use the Latex Math 
%%%%environment, it causes the conversion to HTML complicated. 
%%%Use \emph{} to write math and try to use MATLAB commands for math expressions
%%%All Parameters for the \Block \Port and \Parameter commands must be present.

\Block{Snapshot Capture}{snap}{snap}{Aaron Parsons}{Aaron Parsons, Ben Blackman}{The snap block provides a packaged solution to capturing data from the FPGA fabric and making it accessible from the CPU. snap captures to a 32 bit wide shared BRAM.}

\begin{ParameterTable}
\Parameter{No. of Samples ($2^?$)}{nsamples}{Specifies the depth of the Shared BRAM(s); i.e. the number of 32bit samples which are stored per capture.}
\end{ParameterTable}

\begin{PortTable}
\Port{din}{IN}{unsigned\_32\_0}{The data to be captured. Regardless of type, the bit-level representation of these numbers are written as 32bit values to the Shared BRAM.}
\Port{trig}{IN}{boolean}{When high, triggers the beginning of a data capture. Thereafter, every enabled data is written to the shared BRAM until it is full.}
\Port{we}{IN}{boolean}{After a trigger is begun, enables a write to Shared BRAM.}
\end{PortTable}

\BlockDesc{
\paragraph{Usage}
Under TinySH/BORPH, this device will have 3 sub-devices: \textit{ctrl}, \textit{bram}, and \textit{addr}. \textit{ctrl} is an input register. Bit 0, when driven from low to high, enables a trigger/data capture to occur. Bit 1, when high, overrides \textit{trig} to trigger instantly. Bit 2, when high, overrides \textit{we} to always write data to bram. \textit{addr} is an output register and records the last address of \textit{bram} to which data was written. \textit{bram} is a 32 bit wide Shared BRAM of the depth specified in \textit{Parameters}.
}
